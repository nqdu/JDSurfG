\documentclass[UTF8]{article}
\usepackage{amsmath}
\usepackage{setspace}
\usepackage{caption}
\usepackage{graphicx, subfig}
%\usepackage{noindent}
\usepackage{listings}
\usepackage{xcolor}
\usepackage{enumerate}
\usepackage[colorlinks,linkcolor=blue,citecolor=blue]{hyperref}
% biblio
\usepackage[authoryear]{natbib}

\lstset{
    basicstyle   =   \sffamily,          % code style
    keywordstyle =   \bfseries,          % keyword 
    commentstyle =   \rmfamily\itshape,  % style of comments
    stringstyle  =   \ttfamily,  % style of strings
    flexiblecolumns,                % 
    numbers      =   left,   % location of line number
    showspaces   =   false,  % 
    numberstyle  =   \tiny \ttfamily,    
    showstringspaces  =   false,
    captionpos          =   t,     
    frame               =   lrtb,   % frame
}

\title{\textbf{Manual of JDSurfG}}
\author{Nanqiao Du \\ Version 3.0}
\date{\today}

\setlength{\listparindent}{0pt}
\begin{document}
\maketitle
\tableofcontents
\newpage


\section{Introduction}
This document describes how to use \texttt{JDSurfG} package 
to perform 3-D joint inversion of shear wave velocity
in crust and upper mantle by surface wave dispersion and
gravity anomaly data. This package is consisted of three
independent modules: generating gravity matrix in spherical
coordinates, conducting direct surface wave tomographic 
inversion, and performing tomographic joint inversion of 
dispersion and gravity anomaly data. In joint inversion, we
utilize direct surface wave tomography method \citep{Fang2015}
to compute 3-D (pseudo) sensitivity kernel of surface waves
traveltime, and the adaptive Gauss-Legendre integration 
method \citep{RN35} is adopted to calculate the gravity 
forward matrix. Based on empirical relations between seismic
velocity and density of rocks \citep{Brocher05}, this 
package would be a useful tool to investigate 3-D shear 
wave structure in the crust and upper mantle. For more details, 
please refer to the paper \citep{Du2021} \\

This package is written by \texttt{C++} and \texttt{Fortran}.
\texttt{C++} is responsible for the main logic such as 
handling I/O, providing useful data structures and 
interfaces. And the \texttt{Fortran} part, derived from 
package \href{https://github.com/HongjianFang/DSurfTomo/tree/stable/src}{DSurfTomo}
with several important modifications (by adding 
parallel mode and analytical derivatives), is used to 
compute surface wave traveltimes and corresponding 1D/3D 
sensitivity kernels.\\

I should note that this package was only tested by 
several people on similar computational conditions, 
so it might contain some bugs in it. I'd appreciate 
to receive bug reportings and modify some part of 
it when obtaining good suggestions. In the long term, 
I would add some new functions to this package, 
you could see the \texttt{TODO LIST} in \texttt{README}.

\section{Preliminaries}
This package utilizes \texttt{C++} library \href{http://eigen.tuxfamily.org/index.php?title=Main_Page}{Eigen}
to handle multi-dimensional arrays, and \href{http://www.gnu.org/software/gsl/}{GSL}
to compute Gauss-Legendre nodes' locations and weights. So you need 
to install these two libraries in your own computer before
compiling this package. Also, because some of this package
depend on template expressions, you should make sure
that your \texttt{C++} compiler support \texttt{C++11}
standards (\texttt{GCC} >= 4.8).

\section{Installation}
After you have downloaded the .zip file from Github,
you can unzip it by using:
\begin{lstlisting}[language=bash]
    unzip JDSurfG.zip
\end{lstlisting}
Before preceding to compile this package, you should 
add the Include path of \texttt{Eigen} and \texttt{GSL} 
and the Library path of \texttt{GSL} in 
\texttt{JDSurfG/include/Makefile} if you don't modify your 
\textasciitilde \texttt{/.bashrc} when you install these
two libraies. For \texttt{Eigen}, you could add include path
as
\begin{lstlisting}[language=bash]
    EIGEN_INC = -I/path/to/your/eigen
\end{lstlisting}
For \texttt{GSL} library, a convinient way is to 
use \texttt{gsl-config} in the \texttt{bin}
directory in the \texttt{GSL} Installation path:
\begin{lstlisting}[language=c]
    GSL_LIB = $(shell gsl-config --libs)
    GSL_INC = $(shell gsl-config --cflags)
\end{lstlisting}

Then compile this package by:
\begin{lstlisting}[language=bash]
    cd JDSurfG
    make
\end{lstlisting}
Then you will find four excecutable files in the 
directory \texttt{bin}:
\begin{lstlisting}[language=bash]
    mkmat DSurfTomo JointSG syngrav  
\end{lstlisting}
That means you have finished installation. \\

\section{Direct Surface Wave Tomography Module} 
This module need 3 (4 if checkerboard or other test 
is required) input files:
\begin{itemize}
\item \verb!DSurfTomo.in!: contains information of 
        input model, dispersion data and inversion 
        parameters.
\item \verb!surfdataSC.dat!: dispersion data for 
            each station pair
\item \verb!MOD!: Initial model
\item \verb!MOD.true!: True model (if 
                        additional tests are required )
\end{itemize}
There are some points to note before we precede to the 
format of each file:
\begin{enumerate}[(1)]
    \item Input model is homogeneous in latitudinal and 
            longitudinal direction, but the grid size 
            could vary in depth direction.
    \item The input model should be slightly larger than
          the target one in order to perform \texttt{B-Spline}
          smoothing for forward computation. To be specific,
         if our target model is in region 
         $(lat_0 \sim lat_1,lon_0\sim lon_1,0\sim dep_1)$, 
        you should edit your input model at least in 
        region $(lat_0-dlat \sim lat_1+dlat, lon_0-dlon \sim lon_1+dlon,0 \sim dep_1+dz)$.
    \item Stations should be far away from target model 
            boundaries, or some warnings will be given 
            when you running this package. This is to 
            avoid that surface wave train travels along
            the models' boundaries.
\end{enumerate}
Now there are the formats of every file below:
\subsection{Parameter File} \label{DSurfTomo}
The parameter file is called \texttt{DSurfTomo.in}
in this instruction. It is a self-explanatory file.
Here is a template of this file below:
\\
\begin{lstlisting}
################################
# DSurfTomo Program Parameters 
################################

# input model description:
33 36 25       # nlat nlon nz (no. of points in lat lon and depth direction)
35.3 99.7      # lat0 lon0 (upper left point,[lat,lon], in degree)
0.3 0.3        # dlat dlon (grid interval(degree) in lat and lon direction)

# dispersion forward/adjoint computation parameters
# Insert 'ninsert' points in the grid-based model 
# to compute dispersion curves
3              # ninsert (insert several layers, 2~5)
8              # no.of threads used 

# inversion parameters
20.0 1.0       # smooth damp
2.0 4.7        # minimum velocity, maximum velocity (a priori information)
10             # maximum iteration
2              # no. of threads used in Linear System Solver

# dispersion data description 
16             # kmaxRc (followed by periods)
4.0 6.0 8.0 10.0 12.0 14.0 16.0 18.0 20.0 22.0 24.0 26.0 28.0 30.0 32.0 34.0
16             # kmaxRg (followed by periods)
4.0 6.0 8.0 10.0 12.0 14.0 16.0 18.0 20.0 22.0 24.0 26.0 28.0 30.0 32.0 34.0
0              # kmaxLc (followed by periods)
0              # kmaxLg (followed by periods)

# synthetic test
1              # synthetic flag(0:real data,1:synthetic)
3.5            # noise level, std value of gaussian noise
    
\end{lstlisting}
When you run this program, it will automatically skip
all the empty lines and comments (denoted by '\#'). So you 
could modify this file (by adding your own comments) as long 
as you don't change the order of these parameters. Now let's
see the meaning of each parameter in the template file.

\begin{enumerate}
    \item \texttt{Input Model Description Block}. This block 
    contains information of input model. The first line contains 
    three integers which denote the number of grid points
    in latitude,longitude,and in 
    depth respectively. The second line denotes coordinates 
    of the origin point (NorthWest point), in degree. The last 
    line denotes the grid interval in latitudinal and 
    longitudinal direction, both in degree.
    
    \item \texttt{Forward/Adjoint Computation Block}. This 
    block contains parameters for forward/adjoint computation.
    The first line is an integer we call it "ninsert". 
    As metioned above, we use grid nodes to parameterize 
    our model and thus all physical parameters are defined directly 
    at grid points. However, in order to compute surface wave 
    dispersion by Thomas-Haskell Matrix Method \citep{Haskell53,schwab1972},
    we need to convert grid-based model to layer-based one. In that 
    case, we insert 'ninsert' points between adjacent points to do 
    that conversion. The next integer is how many threads you 
    want to use in calculation of dispersion curves and do 
    ray tracing with Fast Marching Method \citep{rawlinson2004}. 

    \item \texttt{Inverse Parameters Block}. This block contains 
    all parameters used in the inverse problem. The first line indicates 
    the smoothing and damping factor in solving the linear systems.
    Then the followed line is the restriction of your  
    inverted model (the minimum and maximum velocity). The next line 
    is an integer which is the maximum iterations you want to iterate. The
    last line is the number of threads used in solving linear 
    systems, usually 2 is enough.
    
    \item \texttt{Dispersion Data Block}. This block describes 
    the information of your dispersion data. The first line is 
    the number of periods your Rayleigh wave phase velocity
    disperion data spans. If this number is not zero, then you 
    should write all the periods in the next line. Other lines 
    are for Rayleigh group dispersion, Love phase, and Love group 
    respectively.

    \item \texttt{Synthetic Test Blcok}. This block contains all 
    parameters used in synthetic test. The first one is a flag variable
    to denote whether you will conduct synthetic test when using this 
    program. The next line is the noiselevel of gaussian noise, 
    defined as below:
    \[   
        d_{obs}^i = d_{syn}^i +  noiselevel * N(0,1)
    \]

\end{enumerate}

\subsection{Initial Model and True Model File}
\texttt{Please Note that the format of MOD.true in current version 
is different from previous versions!} \\ 

The initial and true model are called \verb!MOD! 
and \verb!MOD.true! respectively in this instruction. 
They are used for initial input model, 
and for synthetic test dataset (like in checkerboard test). 
The format of these two files are listed below: \\

The first line of each files are depths (in km) of each grid points,
i.e. there would be \texttt{nz} numbers in this line. 
The subsequent lines of \verb!MOD! are shear wave velocities.
If we store the velocity with the format V(nz,nlon,nlat)
(Row-major,where $V(0,0,0)$ is the shear-wave velocity of the 
\texttt{top NorthWest} point at \texttt{dep(0)} km, and V(nz-1,nlon-1,nlat-1) is 
the velocity of \texttt{bottom SouthEast} point at dep(nz-1) km), then it 
should be print to MOD as following:
\begin{lstlisting}[language=c++]
//c++
for(int i=0;i<nz;i++){
for(int j=0;j<nlon;j++){
    for(int k=0;k<nlat;k++){
        fprintf(fileptr,"%f ",V(i,j,k));
    }
    fprintf(fileptr,"\n");
}} 
\end{lstlisting}

\subsection{Dispersion Data File}
This dispersion data in \verb!example/! is surfdataSC.dat,
which contains dispersions (distance/traveltime) obtained
surface wave or cross-correlation functions\\
Here we list its first 9 lines:\\
\verb!#!  31.962600 108.646000 1 2 0\\
33.732800 105.764100 3.0840\\
34.342500 106.020600 3.1154\\
33.357400 104.991700 3.0484\\
33.356800 106.139500 3.0820\\
34.128300 107.817000 3.1250\\
33.229300 106.800200 3.0575\\
\verb!#! 29.905000 107.232500 1 2 0\\
34.020000 102.060100 2.7532\\

Obviously, this file contains two kinds of infomation:
surface wave source or master station (lines begin with
\verb!#!) and receiver stations (without \verb!#! in each
line). For a source or master station, the format of 
this line is:\\
\begin{center}
    \verb!#! \texttt{latitude longitude period-index wavetype
     velotype}
\end{center}

\begin{itemize}
    \item \texttt{wavetype}: the type of surface wave, for 
                            Rayleigh it is 2, and it is 1 for Love wave.
    \item \texttt{velotype}: velocity type, 0 for phase velocity
                            and 1 for group velocity.
    \item \texttt{period-index}: The period index number.
\end{itemize}
For example, if in our dataset, the dispersion periods 
for Rayleigh phase velocity is at periods 4s,5s,6s,7s, 
and this line contains information only at 5s, then 
the last 3 numbers 
are \texttt{2,2,0}.\\

Then the following lines contain all the receiver stations
that you have successfully extracted dispersion curves
at this period, this wave type and this velocity type.
And the format of each line is:
\begin{center}
    \texttt{latitude longitude velocity}
\end{center} 


\subsection{Run This Module}
After you have prepared all the required files in a folder,
then you could enter into this folder,and run this command
in your shell:
\begin{lstlisting}[language=bash]
./DSurfTomo -h
\end{lstlisting}
Then you could type in all required files according to the 
order on the screen.

\subsection{Output Files}\label{surface wave output files}
This module will output two (temporary) folders: \verb!kernel/! 
and \verb!results/!. \texttt{kernel} is a tempoaray folder ,
and it contains 3-D pseudo sensitivity kenrels of surface wave
travel times. \verb!results/! contains all the intermediate models 
after each iteration. Here are information about the format
of these files:

\begin{itemize}
\item \verb!kernel/!: 3-D pseudo sensitivity kernel. The number of
txt files is equal to no. of threads used in your program 
(see \ref{DSurfTomo}), and start from 0.
The kernel matrix is Compressed Sparse Row matrix. It looks like:
\begin{lstlisting}
# 0 1283
24292 2.18522
24293 2.10063
24322 2.29456
24323 2.90529
24324 2.77207
24353 2.45755
24354 3.14743
24355 2.99618
24384 2.14853
\end{lstlisting}

Note there are two numbers followed '\#'. The first one is the 
row-index, and another one is the no. of nonzeros in this row
(1283 for current case). Then the following (1283) rows
are column-index and matrix elements. It should be noted that 
this folder is only a temporary folder in the inversion process,
and if will be deleted after each iteration.

\item \verb!results/!:contains models (\verb!mod_iter*!)
and surface wave travel time (\verb!res*.dat!) of each iteration.
The format of model files is:
\begin{center}
    \texttt{longitude latitude depth Vs},   
\end{center}
And the format of traveltime file is (9 columns): 
\begin{center}
    \small{\texttt{distance observation synthetic lat1 lon1 lat2 lon2 
    period wavetype}}.
\end{center}

\end{itemize}

\section{Joint Inversion Module}
This module needs 4-6 files in total:
\begin{itemize}
    \item \verb!JointSG.in!: contains information of input model,
            dispersion data and inversion parameters.
    \item \verb!surfdataSC.dat!: surface wave dispersion data.
    \item \verb!obsgrav.dat!: Gravity anomaly dataset.
    \item \verb!gravmat.dat!: sensitivity kernel of 
                gravity to density.
    \item \verb!MOD!: Initial model.
    \item \verb!MOD.true!: checkerboard model.
    \item \texttt{MOD.ref}: gravity reference model, used 
            for compute gravity anomaly.
\end{itemize}
Here are the formats of each file.

\subsection{Parameter File}
The difference between \verb!JointSG.in! and \verb!DSurfTomo.in! 
is little except last three blocks:
\begin{lstlisting}
# synthetic test 
1             # synthetic flag(0:real data,1:synthetic)
3.5 0.05      # noiselevel for traveltime and Bourguer

# weight parameter 
0.5           # parameter p (Julia(2000)) for 2 datasets, 0-1
3.5 0.05      # std parameters for two datasets 

# gravity processing
# whether to remove average value of synthetic gravity data    
0             # 1 remove, 0 not remove 
\end{lstlisting}

In order to know the meaning of each parameter in \texttt{
Weight Parameters Block}, 
here I list the weighted cost function of joint inverse 
problem \citep{Julia2000}:
\[
    L = \frac{p}{N_1 \sigma_1^2}\sum_{i=1}^{N_1} (d_{1,i}- 
        d_{1,i}^o)^2 +   
        \frac{1-p}{N_2 \sigma_2^2} \sum_{i=1}^{N_2}
        (d_{2,i}- d_{2,i}^o)^2 \tag{1}
\]
where $\mathbf{d}_{1,2}$ is the data vector for each dataset. $N_{1,2}$
is the size of each dataset and $\sigma_{1,2}$ is the corresponding
errors. $p$ is a parameter in the range [0,1] to control the contribution
of each dataset manually. \\

The additional \texttt{gravity processing block} contains only one integer. 
When this integer is equal to 1, the program will automatically 
remove the average value of observed and synthetic gravity data. 
This option is applied to remove large-scale density anomalies 
in the deep mantle. For more details please refer to \ref{Gravref}.

\subsection{Gravity Data File}
The format of this file is quite simple, just print all 
the gravity data to this file. Each line of it contains 
the longitude,latitude and gravity anomaly at this point. \\

Here I list the first 9 lines of this file to give you
an example:\\
100.000000 35.000000 -224.206921\\
100.000000 34.950000 -224.110699\\
100.000000 34.900000 -241.774731\\
100.000000 34.850000 -240.245968\\
100.000000 34.800000 -234.187542\\
100.000000 34.750000 -246.670605\\
100.000000 34.700000 -238.575939\\
100.000000 34.650000 -241.357250\\
100.000000 34.600000 -260.059429\\

\subsection{Gravity Matrix File}
This file is generated by gravity module, please refer to 
chapter \ref{Gravity Matrix}.

\subsection{Gravity Reference Model File}\label{Gravref}
Based on the definition of gravity anomaly, it reflects the 
density perturbance according to a normal elliptoid. Thus we need 
to convert the absolute anomalies to local anomalies in our research 
area. Before inversion, we will remove the average value of 
oberserved anomalies. So in the real data 
invesion, there are 3 steps to compute relative anomalies:
\begin{itemize}
    \item Compute density distribution of 
        current S-wave model and reference model (through
        empirical relations).
    \item Compute density anomaly, then utilize gravity matrix
          to get gravity anomalies.
    \item Remove average value of the synthetic anomaly data.
\end{itemize}
This model could be a user self-defined model, and you could also use the 
default one (the average of initial model in every depth). Please 
remember, when you are tackling real data, to set the integer to 1 in the gravity processing block 
in \texttt{JointSG.in} file. 

\subsection{Run This Module}
After you have prepared all these required files in a folder, 
then you could enter into this folder, and run this command 
in your shell:
\begin{lstlisting}[language=bash]
./JointSG -h
\end{lstlisting}
Then you could type in all required files according to the 
order on the screen.

\subsection{Output Files}
This module will output two temporary folders: 
\verb!kernel/! and \verb!results/!. \\

The format of \verb!kernel/! is the same as the files
in \ref{surface wave output files}. In the folder \verb!results/!,
it contains models (\verb!joint_mod_iter*!), 
surface wave travel time (\verb!res_surf*.dat!) and 
graivty anomaly (\verb!res_grav*.dat!) 
of each iteration. The format of gravity anomaly is:
\begin{center}
    \texttt{lon lat observed-anomalies synthetic-anomalies}
\end{center}

\section{Gravity Module}
This module requires three input files:
\begin{itemize}
    \item \verb!JointSG.in! or \verb!DSurfTomo.in!: contains model information
    \item \texttt{MOD} : only first line of it are needed.
    \item \texttt{obsgrav.dat}: contains coordinates of each 
                     obeservation points.
\end{itemize}
These formats have been illustrated in previous chapters.

\subsection{Run This Module}
After you prepare all these required files in a folder,
then you could enter into this folder,and print in your shell:
\begin{lstlisting}[language=bash]
./mkmat -h
\end{lstlisting}
Then you could type in all required files according to 
the order on the screen.

\subsection{Output Files}\label{Gravity Matrix}
This module output only one file: \verb!gravmat.dat!. 
This is the density kernel for gravity anomaly. It is also
a Compressed Sparse Row Matrix, the format of it is the same
as surface wave kernel in \ref{surface wave output files}.

\section{Advanced Options}

\subsection{Gauss-Legendre Quadrature and 
the Sparcity of Gravity Matrix}
In \verb!src/gravity/gravmat.cpp!, you could modify the 
max and min order of Gauss-Legendre Quadrature,
and the maximum distance beyond which gravity matrix is zero (The attraction between two 
points is negligible if beyond this distance)
\begin{lstlisting}
const int NMIN=3,NMAX=256;
const float maxdis = 100.0;
\end{lstlisting}
Also, because we use a preallocated memory to store gravity matrix 
in the program, you should make sure this memory is large enough. If 
your memory is small, this program will complain and stop with
infomation like \texttt{"Please increase the sparse ratio!"}. In fact,
the size of the memory (bytes) used is from:
\[
    Mem = sparse * m * n * 4 
\]
Where \texttt{m} is the number of observed gravity data, 
\texttt{n} is the total size of your 
model, and \texttt{sparse} is the sparse ratio of the gravity matrix.
To change that memory used, you should modify this line in 
\texttt{src/gravity/gravmat.cpp}:
\begin{lstlisting}
const float sparse = 0.2;
\end{lstlisting}
where sparse should be a floating number between 0 and 1. Note that 
this number is dependent on the \texttt{maxdis} in the same file.
After you change (one of ) these parameters, please remember
to recompile this program. 

\subsection{Empirical Relations}
Empirical relations are utilized to connect density and 
velocity of rocks in our program. If you have better relations 
in your research area, you could change this relation 
in \verb!src/utils/bkg_model.cpp! 
\begin{lstlisting}[language=c++]
void MOD3d:: 
empirical_relation(float vsz,float &vpz,float &rhoz)
{
    vpz = 0.9409 + 2.0947*vsz - 0.8206*pow(vsz,2)+ 
            0.2683*pow(vsz,3) - 0.0251*pow(vsz,4);
    rhoz = 1.6612 * vpz - 0.4721 * pow(vpz,2) + 
            0.0671 * pow(vpz,3) - 0.0043 * pow(vpz,4) + 
            0.000106 * pow(vpz,5);
}
\end{lstlisting}
And if you want to conduct joint inversion, please remember to
change the derivative function in the same file.

\subsection{Random Seed}
In our program, to make sure the generated random number the same
in each independent numerical experiments, we fix the random seed
in \texttt{src/utils/gaussian.cpp}:
\begin{lstlisting}[language=c++]
static default_random_engine e(11);
\end{lstlisting}
You could change 11 to other integer if required.

\section{Tips and Tools}
\subsection{Warnings}
In fact, \texttt{DSurfTomo} and \texttt{JointSG} are all dependent 
on two modules. First one is the forward computation of dispersion 
curves from 1-D model, and another one is the ray-tracing on 
2-D spherical surface. There would be some possible errors in these 
two modules due to incompatible input model format, terrible initial 
model, and even the inappropriate location of your stations. To tackle possible 
problems, the two modules will print some information on the screen 
or in a output file to help you debug.

The first possible warning is \texttt{"WARNING:improper initial value in 
disper-no zero found"} on the screen. Then a \texttt{fort.66} file will 
be generated in the running directory. This error is due to 
a strange 1-D model that the dispersion module cannot find 
dispersion curves for this model. There are some possible reasons for 
this warning. If you find this problem for the first iteration,
this problem may from that the format of your initial model 
is different from the required one. You could check the model in 
file \texttt{fort.66} and compare that with your model. 
Another possible reason is that you take inappropriate 
Tickonov regulariztion parameters (i.e. damping and smooth factors)
which lead to abrupt change of previous model.

The second warning is \texttt{"Source lies outside bounds 
of model (lat,long)="}.
This reason is caused by that the receivers are too close to 
the boundary of the current model that ray paths are along 
this boundary. If this warning happens, please enlarge your 
model to avoid this problem.

\subsection{L-curve Analysis}
Usually L-curve analysis is applied to determine the best 
suitable smoothing and damping factor for a inverse problem. In our 
numerical experiment, we find that the damping factor affects the 
final result smaller than smoothing factor. So we \texttt{recommend}
you to keep the damping factor to a small value (such as 0.001) during 
your inversion, and only tune the smoothing factor. 


In the folder \texttt{utils/}, we have prepared two scripts to help 
you tune the smoothing factor. The first one is a bash script 
\texttt{\text{tune\_L\_curve.sh}}
that will generate a series of inversion results in the folder 
\texttt{storage/}. Please read 
this script carefully and then try your own examples.

The second python script \texttt{utils/lcurve.py} will help you 
generate a L-curve based on your results. Before running 
this script, please change all the 
required parameters in the function \texttt{main()}.  

\subsection{Checkpoint Restart}
Sometimes you want to start from one of the results (like mod\_iter5.dat)
instead of the initial model. We prepared a python script
\texttt{utils/out2init.py}, you could run this script like this 
\begin{lstlisting}
python out2init.py mod_iter5.dat MOD05 
\end{lstlisting}
Then this script will convert the output model to the format 
of the initial model. This script will be useful in joint inversion.
And also, it may help you restart from some checkpoint if some issues 
happend.


% biblio GJI
\bibliographystyle{abbrvnat}
\bibliography{bibliography}
\end{document}
